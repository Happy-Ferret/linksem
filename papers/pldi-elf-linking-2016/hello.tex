\section{Linking \texttt{hello}}

For programming languages, a ``hello, world!'' program is among the simplest possible.
However, for a linker, such a simple program amounts to a very complex job,
since it links with the C library---usually the most complex library on the system,
in terms of the linker features it exercises.

In this section we outline what happens when a hello-world ,

a substantial C library slightly simpler than the system-default GNU C library.

At each point we describe the relevant features of our formalisation

parse command line -- 

parse command line. enumerate inputs, their options, and global link options -- 
    by processing the command line. different options can be invoked
    for different inputs.
    global link options

enumerate objects -- each input item might include multiple objects, 
     might be an object file or linker script, etc.
     ... elaborate away the archive structure, but retain enough information to resolve symbols

resolve symbols -- rules depend on where inputs came from, e.g. archives are different, 
    groups have an effect, etc..

discard unreferenced objects -- using symbol reference as a dependency graph, 
      selecting only those needed in the link

generate support data -- dynamic relocations, GOT / PLT etc.. must happen *before* linker script, 
       so that script can control where they get placed

   -- optimise! simplify relocs etc..

compose output sections -- a pass over the linker script, 
      also generating symbol definitions etc. (PHASE issue)

(optional) discard unreferenced sections

assign addresses -- another pass over the linker script

apply relocations -- now that addresses have been assigned, 

generate output
