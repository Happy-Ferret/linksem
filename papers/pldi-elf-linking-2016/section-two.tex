\section{Understanding linking}

Linking is used to implement separate compilation 
of languages including C, \Cplusplus{}, Fortran, Ada, Objective-C, and
many others.
If linking were only about separate compilation, it would have a natural specification
in terms of these languages' source program semantics.
Previous formal models of linking~\cite{cardelli-program-1997, glew-type-safe-1999, machkasova-calculus-2000, wells-equational-2000, fagorzi-calculus-2007}
have worked on this basis, 
seeking to provide a formal basis for separate compilation with verified safety properties, as has
previous research on linking in the context of verified compilation~\cite{stewart-compositional-2015, kang-lightweight-2015}.

However, linking is not simply a matter of separate compilation.
Systems code and application code alike 
use linker features to achieve effects that are \emph{unrelated to} separate compilation,
which mostly \emph{cannot be expressed} in the relevant source language(s), 
and which, in some cases, \emph{actively break} the na\"ive semantics of the source language.
In this section we survey these features,
consider how they are (or aren't) currently specified, 
and consider the consequences for 
how toolchains should be specified and what services they might provide.
%motivate a different approach to specifying linking.


% For example, Cardelli formalised a system of ``linksets'' 
% in which ``separate compilability'' could be formally stated and checked,
% modelling linking as a property-preserving merge operation,
% and assuming a source-level interface formalism including notions of signature and type checking.
% 
% Although such work can be valuable in new language and toolchain designs, 
% it does not suffice for modelling existing code.
% 
% Currently we are aware of no formal semantics for any substantial fragment of linker-speak,
% making only informal reasoning possible.
% 
% Existing approaches to formally verified toolchains 
% have effectively replaced the linker
% with a minimal language-specific linker
% intended merely to provide separate compilation semantics, 
% therefore not addressing the features on which many real codebases rely.


\subsection{Linking basics}

In this paper we focus on the linker-speak of System~V-derived Unix environments,
including GNU/Linux and all modern BSD variants.
Windows and Mac OS (Darwin) environments exhibit broadly similar feature sets,
but the System~V environment's is generally a superset.

Following \citet{gingell_1987_shared}, we say ``batch linker''
to mean the \textsf{ld} tool run at build time, and 
``dynamic linker'' to refer to the \textsf{ld.so} dynamic linker running
at load time.

A batch linker takes multiple modules of \emph{relocatable object code} 
as input (in the form of \textsf{.o} files, or groups thereof bundled into 
\textsf{.a} ``archive'' files).
It produces a single ``linked'' binary as output.
Linked binaries are usually executables, and embody the input modules.
The main work of the linker is to organise the input modules' content
in a single logical memory image, 
and to turn symbolic references (encoded as metadata in the input files), 
into concrete bindings encoded as bit-patterns in the output binary.

Object code consists of \emph{sections}, \emph{symbols} and \emph{relocations}.
Sections are chunks of bytes, treated as indivisible units
by the linker: they may be combined or rearranged, but not broken apart. 
Symbols give names to particular ranges of a section, 
corresponding to source-level definitions
(functions or global variables, say).\footnote{ELF symbol names
are arbitrary sequences of non-zero bytes, so can encode 
essentially any source-language identifier verbatim. However, 
most language implementations generate a lexically narrower set of symbols,
according to the lexemes accepted as symbols by the assembler.
This explains why name-mangling schemes are used by \Cplusplus{} compilers: 
\Cplusplus{} punctuation such as `\textsf{::}' is not assembler-friendly. C identifiers 
need not be mangled, by design.} 
Some symbols are instead \emph{undefined},
meaning they give a name to a definition which does not exist 
in that object file. 
Rather, it exists in a different object file that is to be linked in.
References to symbols are encoded as \emph{relocation records}, sometimes called 
``fixups'': they describe how 
the section's bytes must be fixed up to point to the intended referent.
Each relocation record specifies a ``type'' of relocation, corresponding to 
different instruction encodings and pointer encodings.
Even if a symbol is defined locally in a given object file, 
references to it might still require relocation.
This is because the symbol's address is not yet bound.
For example, a pointer called \textsf{p} 
initialized to point to a specific global variable \textsf{x}
will appear in object code as
(1) a word-sized portion of some section (probably called \textsf{\.data});
(2) a defined symbol called \textsf{p} marking that range within the section; 
(3) a relocation record marking that range as pointing to \textsf{x},
whose address is not yet known.
The symbol \textsf{x} also exists in the object file; it 
may be either a defined or undefined symbol, 
depending on whether it was defined in the originating compilation unit.
In either case, its address is not decided until during linking, so 
\textsf{p}'s relocation record remains essential.
Making a concrete binding is called \emph{applying relocations}, 
betraying its intertwinedness with the process of assigning memory addresses,
and the metadata describing references are called \emph{relocation records}.
The actual bytes of \textsf{p} will be filled with an arbitrary value,
typically zero; only during relocation will they be filled in.

%These bit-patterns are either the address fields of encoded machine instructions (in code), 
%or pointer representations (in data).

Having composed a memory image out of its inputs, 
and applied relocations, 
the linker finally outputs a serialised form of this memory image
into the output binary.
If the output binary is completely self-contained, we say it is ``statically linked''. 
Alternatively, modern batch linkers also allow 
some dependencies to be deferred until a later \emph{dynamic linking} step at load time.
In these cases, the linker's input also includes dynamically-linkable libraries.
These are \emph{not} embodied in the output; rather,
at link time they serve as a proxy identifying which libraries will be available at load time.
Any references to them are deferred until load-time binding,
are left unrelocated and output as metadata in the executable.
Shared libraries themselves are also produced by the same batch linker.

Internally there are few differences
between a dynamically-linked executable and a dynamically-linkable shared library.
The principal difference is that the shared library may not 
embed any absolute memory addresses, since it may be loaded
at any address, whereas executables are always loaded from 
address zero.\footnote{Modern linkers can also produce executables
with this property, called ``PIE'' for position-independent executable.}

Loading is the process of actually 
constructing the memory image that the linker 
represented in the output binary in serialised form.
A modern Unix operating system typically contains two loaders: 
one in user-space used for dynamic linking,
and one in the kernel, used for statically linked binaries
and to load the dynamic loader itself.

Dynamically linked binaries, whether executable or library, 
differ from statically linked ones
in that they include metadata used by the dynamic linker
to apply the remaining relocations.
The other key difference is that 
shared libraries, in order for their memory to be physically 
shared at run time across multiple processes, 
are usually crafted so that their text does not need any relocation at all.
Rather, code-to-code and code-to-data references are indirected 
via more compact \emph{non-}shared support structures 
(respectively the GOT, or global offset table, and PLT, or procedure linkage table).
This allows multiple processes to simultaneously map the same unmodified text segment 
at different load addresses.
(Some non-Unix systems avoid the need for position-independent libary code by pre-registering 
certain address ranges, to be used by all processes; Windows DLLs use this scheme.)

\subsection{Linkers and module systems}

A typical programming language ``module system'' provides mechanisms 
dealing in both abstract and concrete modules. 
A module may be abstract in several ways: 
referring to other modules and/or their definitions;
by hiding its internal names from other modules;
by hiding other internal information (e.g.\ representation types) from other modules;
by being parameterised by other modules (a functor) 
     or of definitions they must supply (e.g.\ types they define).
Modules are named definitions,
and ensembles are formed by naming modules, instantiating their parameters,
and (in the case of generative modules) instantiating the modules themselves.

By contrast, linking is almost entirely concrete.
Linker modules, namely files, 
are abstract in only two ways: by referring to other modules,
and by hiding certain named definitions from other modules.
Linker modules themselves are not named definitions for most purposes.\footnote{The 
two exceptions are: diagnostic purposes---since linkers will use 
module filenames to report problems to the user---and memory layout purposes,
since linker scripts are allowed to select input files by name.
Unmatched input files' sections are still included in the link, however, so
this mechanism only allows their placement to be varied, and does not affect
how symbols are resolved.}
The programmer has no direct control of which module gets wired to which other.
Rather, bindings are formed entirely according to symbol names,
without regard to the name of the module (file) supplying them or requiring them.
Linkers have no notion of module parameterisation, nor of module instantiation.
The linker is responsible only for creating the concrete ``wiring'', or binding topology, 
of the output binary, but not for instantiating
Its role in ``programming in the large'' is essentially that 
of module interconnection \citet{deremer_programming_1975}.
FIXME: cite Knit (Reid et al) also.

% \begin{itemize}
% 
% \item abstraction by 
% \item abstraction by parameterisation of units of code (often type-parameterisation; 
%       module-parameterisation, in the case of functors);
% 
% \item concretion by parameter-instantiation;
% 
% \item concretion by 
% 
% \item concretion by binding
% \item concretion of abstract modules into specialised instances;
% 
% \item creation of ensembles by binding together abstract \emph{or} concrete modules.
% 
% \end{itemize}
% 
% Linking relates only the first and last of these.
% It is entirely concrete; it does not instantiate, abstract or specialise.
% Rather, the linker's caller (whether user or compiler) is assumed to have done these already.
% 
% Linking does, however, involve hiding of names---since its primary mechanism
% is using name-matching to bind together ensembles of concrete modules.


\subsection{Linker-speak}

Linker-speak consists of a collection of notations,
which collectively can be thought of as a separate programming language
in which part of every compiled system is expressed.
We call this language ``linker-speak''.
We survey various examples shortly; in brief, the notations
break down as follows.

\begin{description}

\item[Arguments] ---these are command-line options supplied when invoking the linker.
In the case of dynamic linking, environment variables serve the analogous purpose.

\item[Scripts] ---most batch linkers embed a script language, used pervasively.
Scripts come in two flavours: \emph{control scripts}, 
of which there is exactly one in any link job,
and ``implicit'' scripts. 
Implicit scripts are supplied on the command line as if they were object files, 
and are written in a small subset of the linker script language.
Typically they are used to create simple proxy-like object files, defining symbols
and/or sections in a textual rather than binary notation.

\item[Metadata] ---object files contain symbols, sections, relocation records
and other metadata, on which link semantics crucially depend. Many of these 
object file features have corresponding forms in assemblers (``directives'' or pseudo-ops) and compilers (attributes) allowing the programmer
to control the metadata in the assembled or compiled object file.

\item[Conventions] ---applications of linking attach specific meaning to 
specific ways of using symbols, sections and other linker features, even though, 
as a rule, these are formally meaningless to the linker.
For example, ABIs often attach meaning to specific section or symbol names,
require the presence of certain named sections, and so on.
(Additionally, a few linker features flout the rule, by attaching meaning to section names.)

% orphan placement is the big example

\end{description}

\sk{Responding to PS ``we should be clear which of these we cover''---we cover all 
of them partially; the middle two more than the outer two.}

Our work provides a formal specification
of large subsets of all of these notations, with particular emphasis on
the second and third.
Specifically, we focus on the linker-speak of the AT\&T System~V linker
and its descendents. 
In the following, any linker options or script syntax
refer to that accepted by the GNU BFD linker
(still the 
default on both GNU/Linux and BSD environments)
which was based on the AT\&T System~V linker.
\ps{The section below is (probably correctly) organised by
  use-cases, not by linker-speak features.  Maybe make each paragraph
  call out the kind of linker-speak mechanisms it involves?  Or a
  separate section that is organised by mechanism??}
\ps{On a related note, none of these talk linker script explicitly, or
  focus on the expressiveness of the linker-speak mechanisms}

% Sound reasoning about code in any one language
% must account for the changes or relaxations that possible link-time effects might allow.
% Sound reasoning about composed programs 
% must account for the link-time effects that are in place.

% The basic mechanisms of a linker have clear analogues in these source languages.
% Input objects correspond to source files or ``translation units'';
% concatenation of binary sections takes the place of concatenating source;
% binding of symbolic references implements references among these separate source files.
% Symbol namespacing features reflect the top-level scoping rules 
% of C and Fortran, such as \textsf{static} and \textsf{extern} in C.
% 
% Link-time merging of duplicated outputs 
% is a direct consequence of features such as (in Fortran) \textsf{COMMON} blocks, 
% (in C) ambiguous declaration/definitions,  
% or (in C and \Cplusplus{}) % address-takeable 
% inline functions, 
% which may be compiled zero or more times and must then be deduplicated.

% (Interestingly, \Cplusplus{}'s run-time type information and virtual functions 
% do \emph{not} require these functions.)

% concatenation of binary sections;
% binding of references;
% local/global (static/extern) visibility 
% merging duplicated (vague) compiler outputs, e.g. to implement C++ RTTI


% A less obvious
% not only federated
% Federating implementations of one language:
% 
% Providing a separate linker tool 
% allows
% different compilers (or significantly different versions of the same compiler)
% 
% - separate tool allows multiple compilers to coexist;
% in our 
% - conventions allow multiple compilers to cooperate
% 
% Interoperating across languages
% - linker is a language-agnostic mechanism
% - ... but relies on language implementations to use it
% - ... and to adopt conventions that allow meaningful cooperation
% 
% Since linkers lack knowledge of these conventions, 
% and compiler output often does not encode them explicitly,
% this mechanism is potentially fragile,
% with correctness entirely dependent on the well-matchedness of opaque conventions.
% However, it is successfully employed in every mainstream software distribution, 
% (which is seldom made up of binaries generated by the exact same compiler).
% 




% perhaps divide:
% "used by the compiler implementer";
% "used by the end programmer"

% linker features that "often" (\approx "in C") do not correspond to language features

% Systems codebases are shot through with \emph{linker-speak}---linker scripts, 
% linker command-line options (invoked from makefiles), 
% and source code which invokes linker features using inline assembly directives, 
% compiler-specific attribtes or other extensions.

\paragraph{Memory placement}
Systems code often needs to reside at particular places in memory.
For example, typical Unix kernels occupy the higher portion of the address space.
This is specified by a combination of linker scripts and section name attributes. 
A compiler or assembler exposes a mechanism for a definitions to be placed in named sections,
while the linker script allows sections to be assigned to addresses, 
and/or to be ordered relative to one another.
For example, on the PA-RISC architecture, Linux uses the following 
linker script to enforce a particular relative ordering of page table data.
A struct in C cannot be used here, because the definitions must be addressable
symbolically from assembly language as well as from C code.
FIXME: show the preprocessed version, if it's simpler / more snippable.

{\scriptsize\begin{lstlisting}[language=C,columns=flexible,basicstyle=\sffamily]
/* gcc doesn't actually guarantee that global objects will
 *  be laid out in memory in the same order as the order of 
 * declaration, so put these in different sections and use
 * the linker script to order them. */
pmd_t pmd0[PTRS_PER_PMD] __attribute__ ((
    __section__ (".data..vm0.pmd"), aligned(PAGE_SIZE)));
\end{lstlisting}}

{\scriptsize\begin{lstlisting}[language=C,columns=flexible,basicstyle=\sffamily]
/* Put page table entries (swapper_pg_dir) as the first thing 
 * in .bss. This will ensure that it has .bss alignment (PAGE_SIZE).
 */
        . = ALIGN(bss_align);                
        .bss : AT(ADDR(.bss) - LOAD_OFFSET) {
                   *(.data..vm0.pmd)
                   *(.data..vm0.pgd)
                   *(.data..vm0.pte)
                   *(.bss..page_aligned)     
                   *(.dynbss)                
                   *(.bss)                   
                   *(COMMON)                 
        }
\end{lstlisting}}

Even though memory placement is not expressible in source languages, 
code often depends on it---for example, kernels recognise their own addresses
by testing whether they encode a negative signed value.
Control of memory placement is useful not only semantically, 
but as an optimisation, such as 
reordering inputs to improve spatial locality \citep{orr_dynamic_1994}.

% Could "cite" Glek's demand paging init reordering blog post here

\paragraph{Encapsulation}
Hiding implementation details is often achieved using linker features. 
Source-language encapsulation features,
such as C's top-level \textsf{static} modifier, 
map directly to linker features, such as ELF's local symbols.
In this way, intra-module references can still be fixed up by the linker,
in terms of local symbols, but those symbols 
will not be used to form cross-module bindings---encoding the range of top-level
visibility semantics expressible in languages like C, Pascal, Fortran etc..
\ps{is this visibility control uniformly of global variables and of
  code pointers?  presumably not of anything else (e.g.~type names, as
  they're just not present?  (visibility of dwarf info??))}
\sk{ITYM ``code'' not ``code pointers''. But yes, it is uniform. 
There's no separate way in ELF to say ``this is a type''; 
you just have to say it's data.
It then becomes an encoding issue, layered on top of ELF: 
whether ``types'' come out as symbols depends on the language implementation.
Type names do have linkage in \Cplusplus{} iff they have virtual member functions, 
because they correspond to compiler-generated data and code. 
In ordinary C they don't come out at all, because the compiler erases them.
In a liballocs-extended C toolchain, they do, because we retain them at run time.
DWARF info does not take the form of ELF symbols, so no. 
But STABS debug info does, so yes! etc.}
However, linkers expose at least three other encapsulation facilities
that are \emph{not} supported in this way: (1) ELF symbol visibility attributes,
allowing names to be scoped at the coarser granularity of 
dynamically-linked binaries (instead of single object files);
(2) archives,
since inclusion in an archive restricts an object's visibility 
to other modules in the link;
(3) dynamic export control (\textsf{--export-dynamic}), which limits 
which definitions are available for binding or interposition by the dynamic linker.
In addition to the linker command line, 
compiler options (such as \textsf{gcc}'s \textsf{-fvisibility=hidden})
can allow hiding more definitions,
which can lessen a library's dynamic linking overheads
(Drepper~\citet{drepper} advises using it as a matter of course).
Unfortunately this need not be safe in general: \textsf{-fvisibility=hidden}
breaks the source semantics of \Cplusplus{} code if it throws exceptions 
out of the library (REFERENCE).
In general, linker features operate neither wholly above nor wholly below the language level; 
their use may affect both user code and language implementation internals.
Therefore, at present, using them correctly requires developers to understand both.
% so pushes additional complexity onto the user.

% encapsulation (namespace management, in-archive versus not; DSO visibility; export-dynamic)

\paragraph{Build-time substitution}
Link-time mechanisms may be used to substitute one definition for another.
For example, the link semantics of archives allow a C program
to supply its own \textsf{malloc.o} while still linking with 
the remainder of the C library archive (\textsf{libc.a}).
For example, a common performance optimisation is to supply a 
\textsf{malloc()} implementation tailored to the program's allocation behaviour; 
build-time substitution makes is possible to use the remainder of \textsf{libc.a} 
without explicitly creating a copy lacking \textsf{malloc.o} or similar.
% The \textsf{malloc.o} definitions substitute for those in the library.
Outside of archives, multiple definitions are generally not allowed,
but a linker option (such as \textsf{-z~muldefs}) can relax this, 
causing earlier definitions to take precedence over later ones.
Multiple definitions may also occur if all but zero or one is marked ``weak'';
an ordinary ``strong'' definition, of which there can be at most one, always takes precedence.
(We'll see example uses of this shortly, when discussing 
``optionality'' and ``topology alternatives''.)

\paragraph{Load-time substitution (``overriding'')}
Substitution is often used during dynamic linking using the \textsf{LD\_PRELOAD}
feature. This environment variable can be used to supply replacement
definitions, analogous to the \textsf{malloc} example, but for dynamically bound references.
This can create unwanted dependency on change-prone binary-level details\ps{what follows doesn't exactly say
  that. Instead ``Care is needed when using this''?}\sk{``Care'' is too weak: 
there's no way to robustly use this feature without fixing
the binary-level details of your C library; e.g.\ it can easily break on libc upgrades.
I have strengthened it a bit.}, 
since the split between what is statically linked and what is dynamically linked
is an implementation detail of the library.
If it changes, dynamic substitutions may suddenly fail.
For example, \textsf{stat()} in the GNU C library is statically linked even when
dynamic-linking the remainder of the library. This is achieved by having \textsf{libc.so} be the following linker script rather than a shared library.
As a result, dynamic substitution of \textsf{stat()} is not possible.

{\scriptsize\begin{lstlisting}[language=plain,basicstyle=\sffamily,columns=flexible]
OUTPUT_FORMAT(elf64-x86-64)
GROUP ( /lib/x86_64-linux-gnu/libc.so.6 
  /usr/lib/x86_64-linux-gnu/libc_nonshared.a
  AS_NEEDED ( /lib/x86_64-linux-gnu/ld-linux-x86-64.so.2 ) )
\end{lstlisting}}

Linker options can be invoked to bind references earlier than would otherwise happen;
This can avoid indirection, hence improve performance; \textsf{-Bsymbolic} 
requests this when generating System~V shared libraries.
But this also removes load-time substitutability.
Our opening example, which appears to show a def-use edge
from one function to another in the same file, 
was showing that this edge may be subject to substitution.
Dynamic linking substitutes at the granularity of symbols
(a single definition may be substituted),
whereas batch linking substitutes at the granularity of input objects
(only whole compilation units are substituted),
meaning that if our example were in a single source file, it typically
would be substitutable at load time but not at batch link time.
\ps{is this something that mixin module systems can typically do? not sure...}
\sk{Mixins are more about interposition-as-extension, i.e. the next paragraph,
than about pure substitution.
But the latter is a degenerate case of the former---don't use the mixed-in definition!---so
I'd say so.}
However, finer-grained link-time mechanisms could 
also break this edge, because substitution is a degenerate case of \emph{interposition}.

\paragraph{Interposition}
Interposition can be thought of as substitution where the prior definition
is re-used by the substituted one---typically for instrumentation.
(FIXME: say something about infinite regress, here or at the fprintf example?)
Different mechanisms allow this:
at batch link time, the \textsf{--wrap} option exposes a wrapped symbol
with the prefix \textsf{\_\_real\_}.
For example, a program linking in its own \textsf{malloc} 
could be linked with \textsf{--wrap} \textsf{malloc} to profile memory allocations.
The link-time bindings semantics of \textsf{--wrap} are particularly complex because they depend on how the compiler/assembler has mapped source-level definitions and references onto linker-level sections, symbols and relocation records.
This can be influenced by compiler/assembler options, 
source-level attributes, internal implementation decisions, and so on.
For example, when using \textsf{gcc}, compiling with \textsf{-ffunction-sections} results in exposing more bindings to \textsf{--wrap}.\sk{I'm being a bit fast-and-loose here... not really true as stated currently.}
At load time, \textsf{LD\_PRELOAD} can be used for interposition by dynamically looking up the definition that is being substituted for (using \textsf{dlsym()}), and delegating to it. A program dynamically linking against \textsf{malloc} would need to use this trick for its allocation profiler. 

% The latter allows a single binary to be run with or without a particular instrumentation or modification---a trick exploited by 
% instrumentation tools (like early versions of Valgrind), 
% run-time extensions (\textsf{socksify}, \textsf{fakeroot}, \textsf{flcow}),
% compatibility wrappers and the like.
% Meanwhile, 

% Whether or not build-time substitution can affect this binding
% is entirely dependent on the compiler: typically the reference
% would be encoded by a single defined symbol and a single relocation record, 
% ``pre-bound'', hence not substitutable.
% However, it is completely valid ELF to instead encode it as a pair of like-named symbols, 
% one defined and one undefined, in which case link-time substitution could be performed.

\ps{do we need to talk somewhere about the relationship between
  source-language names and ELF-file symbols?}
\sk{Added a footnote some way above. It feels like overkill though.}

% options linker options, 
%compiler/assembler options, and compiler/assembler internals.
%For example, compilers typically 
%e are distinct and complex, since each one 
%interposes on different subsets of binding edges,
%on \textsf{gcc}, 

%how source-level functions are mapped to sections, hence on 
%which references are interposable.

% affect the source-level effects of using \textsf{--wrap} or \textsf{LD\_PRELOAD}.
% %Its semantics are complex because it affects
% %only undefined references
% 
% Exactly which references are available for interposition by \textsf{--wrap}
% depends on decisions in the compiler (or assembler).
% Typically a def-use edge within a single object
% may not be interposed on, although it is possible to emit the symbol table
% such that it can be (REF to liballocs).
% Likewise, a def-use edge in a single text section is typically invisible
% to the linker (since it does not require \emph{relocation}), but
% compiler options (such as \textsf{-ffunction-sections}) 
% can shrink section boundaries to a single function, 
% allowing any function's external references to be interposed on.
% Overall, 



% mention the preload liballocs problem here?)

% substitution (symbol provided both in archives + .o; symbol provided in multiple .a or multiple .so; -z muldefs)


% Although \textsf{--wrap} and \textsf{LD\_PRELOAD} can be used for substituting alternative
% definitions, 
% they are more often used to instrument existing code.
% In both cases,
% a mechanism exists for the substitute code
% to call out to the code that would originally have been bound in 
% (respectively by providing a \textsf{\_\_real\_}-prefixed symbol,
% and by allowing a dynamic lookup using the flag \textsf{RTLD\_NEXT}).

\paragraph{Optionality}

\emph{Weak symbols} allow codebases to reference optional features. 
Unresolved weak symbols are specified to take the value 0, so 
the absence of a definition can be identified by a null pointer test.
In practice, code like the following (from the GNU C library's 
\textsf{freopen()}) is commonplace.

{\scriptsize\begin{lstlisting}
if (&_IO_stdin_used == NULL)
{
  /* do something... */
}
else /* ... */
\end{lstlisting}}

According to the C11 standard, the address-of operator 
always returns a pointer to an object or function
(\S 6.5.3.2 pt 3), which is necessarily distinct from the null pointer
(\S6.3.2.3 pt 3).
However, in order to exploit weak symbols, real implementations
cannot assume this.

% % But weak symbols---a linker feature---allow address-of to yield a null
% % pointer. 
% The C language specification implies that the address-of operator
% in a well-defined program cannot yield a null pointer.
% % (Real implementations of C, of course, know differently.)
% 
% % Linking can
% % express behaviour that the source code didn't, such as which memory
% % addresses a given definition should reside in. Linking can also
% % effectively \emph{change} the source-level program.
% % For example, it can (and does) binding a reference to a definition of a different name. 
% % Moreover,
% % linker features in practice sometimes entail behaviours that are
% % \emph{inconsistent} with source language semantics. 
% % For example, 

\paragraph{Uniqueness and deduplication}
Some programs depend on global uniqueness properties. 
For example, in \Cplusplus{}, two pointers to the same function
must always compare equal.
When header files include inlineable functions that are address-taken, 
implementing this becomes difficult because the
out-of-line code is repeated in multiple compiler outputs.
This is solved by using linker features 
to deduplicate these multiple copies, ensuring a unique definition.
Recent ELF standards call this feature \emph{section groups},
although it is sometimes called ``link once'' (after an earlier GNU extension) or
``COMDAT'' (after the equivalent Windows linker feature).
This facility for enforcing link-time uniqueness 
is a general mechanism, allowing the 
the useful optimisation of tunring pointer equality to be used instead of value equality 
(of immutable objects),
and is exploited by some user-level libraries \citep{kell_towards_2015}.

% dchisnall bug: RTLD_LOCAL can break this


\ps{slightly related: is there some linker mechanism for making copies
  of a bunch of functions \emph{together with their local state}?}
\sk{In short, no. Am guessing ``local state'' means associated static-storage variables? 
In general, linkers turn 1 input item into 0 or 1 copies in the output, never more.
If you want this, it's your job to copy the input files yourself, then link all the copies.
And you probably then have to rename the symbols yourself, too. 
Tools like objcopy are used for this.
This is covered in the ``module system'' text.
}

\paragraph{Aliases}
In standard C, a function has exactly one name. 
At link time, however, the same range of bytes may have multiple symbol names, 
each denoting the same address but with different metadata.
Compilers typically expose this functionality using attributes.
For example, a function \textsf{f} can be marked with the attributes \textsf{alias("g")}
meaning it also has the linker name \textsf{g}.
(We'll see a practical use case for this shortly, in ``topology alternatives''.)
Compiler attributes often encode details crucial to a program's intended meaning,
but their semantics are in effect defined only in terms of linker features.
% This means that two pointers to a differently-named functions
% may still compare equal.
We encountered a bug in the CIL~\cite{necula-cil-2002} C translator, which mis-compiles the \textsf{alias} attribute
(offered by GNU and Sun/Oracle compilers)
by duplicating the function body in a separate definition.
Although C does not define aliasing, 
it has a clear meaning at the linker level: for any given definition, all aliases should have the same address.
No widely-read document happens to specify it.

%   let rt, formals, isva, _ = splitFunctionType vtype in
%   if isva then E.s (error "%a: alias unsupported with varargs."
%                       d_loc !currentLoc);
%   let args = Util.list_map 
%                (fun (n,_,_) -> A.VARIABLE n)
%                (argsToList formals) in
%   let call = A.CALL (A.VARIABLE othername, args) in
%   let stmt = if isVoidType rt then A.COMPUTATION(call, loc)
%                               else A.RETURN(call, loc)
%   in
%   let body = { A.blabels = []; A.battrs = []; A.bstmts = [stmt] } in
%   let fdef = A.FUNDEF (sname, body, loc, loc) in
%   ignore (doDecl true fdef);
%   (* get the new function *)
%   let v,_ = try lookupGlobalVar thisname
%             with Not_found -> E.s (bug "error in doDecl") in
%   v.vattr <- dropAttribute "alias" v.vattr


\paragraph{Topology alternatives}
Aliases are often combined with substitution, visibility and optionality (weak) features 
to yield output objects which 
form different link graphs in different link contexts.
The GNU C library's \textsf{fprintf()} implementation
exhibits this (shown after preprocessing and lightly edited).

{\scriptsize\begin{lstlisting}
/* Write formatted output to STREAM from the format string FORMAT.  */
/* VARARGS2 */
int
__fprintf (FILE *stream, const char *format, ...)
{
  /* snip */
}
extern __typeof (__fprintf) fprintf __attribute__ ((alias ("__fprintf")));
/* We define the function with the real name here.  But deep down in
   libio the original function _IO_fprintf is also needed.  So make
   an alias.  */
extern __typeof (__fprintf) _IO_fprintf __attribute__ ((weak, 
   alias ("__fprintf")));
\end{lstlisting}}

The effect of the two aliases is that
local code which wants to be sure of calling the local definition
(perhaps because it consumes private state, or just to avoid the overhead 
of calling to an object further away) 
can use the name \textsf{\_\_fprintf}.
The standard name \textsf{fprintf} is also provided;
if a substitute is provided by the user 
(much like the \textsf{malloc} substitution we considered earlier),
this will not affect local calls to \textsf{\_\_fprintf}.
Similarly, the alias \textsf{\_IO\_fprintf} is defined
for the benefit of the \textsf{libio} subsystem: 
depending on the build,
this may or may not supply its own definition of \textsf{\_IO\_fprintf},
so the definition is made weak.

% %0000000000054620 g    DF .text  000000000000008f  GLIBC_2.2.5 fprintf
% %0000000000054620  w   DF .text  000000000000008f  GLIBC_2.2.5 _IO_fprintf
% %0000000000054620 g     F .text  000000000000008f .hidden __GI_fprintf
% 
% 
% 
% this is generally not used.
% Instead, one of two aliases is used: 
% a ``hidden'' alias named \textsf{fprintf}, only visible
% within the local object, 
% which means calls within \textsf{libc.so} will take a faster but non-interposable binding 
% (``PLT bypassing'');
% and a weak alias, \textsf{\_IO\_fprintf},
% exposed to the \textsf{libio} subsystem
% that will be linked into the same .
% A different version of \textsf{libio} might supply its own 
% \textsf{\_IO\_fprintf},
% in which case other calls 
% without 
% Because this 
% 
% - the initial definition is named \textsf{\_\_fprintf()}.
% Calls within \textsf{libc.so} which call \textsf{\_\_fprintf()} are guaranteed wish to call 
% the private \textsf{\_\_fprintf()}
% 
% Even if the user supplies their own \textsf{fprintf}, 
% \textsf{libio} will continue to use this definition
% (perhaps because it consumes private state attached to the \textsf{FILE} object), 
% and if the user does not supply their own,
% the weak \textsf{fprintf} alias takes that place.
% 
% in different ways 
% achieve more complex topology control
% , 
% in te presence of s
% , by combining aliases with substitution/visibility/weakness
%            "local code linking against X gets [[P]];
%             remote code linking against X gets its own X if it has one, else...;
%             local code linking against __X gets the user's (override) for X if it has one, else..."
%     ... we could call this "selective substitution"

\paragraph{Introspection}
Linker features are used to allow programs to introspect on their own structure.
The \textsf{end}, \textsf{etext} and \textsf{edata} symbols 
allow programs to test whether a pointer falls within the 
executable's data or text segments. 
This is used variously in profiling code, garbage collectors,
other dynamic analyses, diagnostic pretty-printing 
(e.g.\ printing a pointer-to-data differently from a pointer-to-text)
and so on.
Dynamic linkers offer a richer interface 
in terms of \textsf{dlsym()} (name-to-address) and \textsf{dladdr()} (address-to-name) functions.
Some systems code relies on the ability to introspect its own structure,
often during initialization.
%acquiring reference at link time 
%where being passed a reference at run time is not possible or not efficient.
The GNU C library's static-linking initializers 
use specially placed symbols to initialize a table (the procedure linkage table; see \S\ref{sec:})
at start-up.
For similar reasons, the Linux kernel 
makes use of a GNU linker extension in which \emph{any}
section whose name could be a C identifier (here \textsf{\_\_ksymtab})
is given \textsf{\_\_start\_}- and \textsf{\_\_stop\_}-prefixed
marker symbols.
Meanwhile, any typical (SunOS-style) dynamic linker, despite being written in C,
must first \emph{introspectively relocate its own data and text}, 
meaning its code may use only PC- and stack-relative addressing
until the pointers and absolute jumps in its own image
have been fixed up.
%\footnote{This makes its behaviour completely undefined
%as far as the C standard is concerned, yet dynamic linkers 
%are seldom implemented in any other language.}
Its (batch) linker script defines a special symbol, typically \textsf{\_begin}, 
that it (the dynamic linker) uses to determine its own load address.
Although all these codebases are superficially 
programs written in C, reasoning about them
without semantic knowledge of linking features 
is at best imprecise and at worst unsound.

% is impossible Programs making use of linker-supplied definitions 
% may assume they exist, and with certain semantics, 
% to which source-only analyses are oblivious.

%On architectures with PC-relative data addressing,
%the following code suffices
%
%	void *base_addr = &_begin;
%	bootstrap_relocate(base_addr);
%
% ... but not on others

% se definitions, even though source semantics say they shouldn't exist

\paragraph{Versioning} 
Shared libraries must allow old clients to be executed against a newer
library binary.
To prevent interface changes from breaking old clients, 
modern dynamic linkers support symbol versioning---allowing
multiple versions of an interface to be exposed by a single
backward-compatible binary.
It follows that source code wanting to supply ``a definition''
for \textsf{fopen} now must be able to supply several.
Programmers achieve this using assembler directives, 
rather than language extensions.
(Interestingly, linking \emph{multiple versions of a library} independently
in the same process is effectively unsupported by ELF dynamic linking.)
%unless their symbol namespaces are completely disjoint;
%the multiple versions must be packaged into a single object with version metadata.
Versioning interacts with introspection: symbol names
are no longer enough to identify a unique definition,
so to avoid ambiguity, symbol versions must be supplied (using the \textsf{dlvsym()} call).

\paragraph{Other linker features}
Linker-speak has various other roles. (FIXME: tidy up)
It allows communicating with the loader, via custom program headers,
such as memory-protecting additional regions of the address space
(a GNU extension); 
communicating with profiling and debugging tools; 
configuration for the deployment environment 
(setting library paths, the loader or ``interpreter'' path, and similar); 
optimisations for code size 
(\textsf{--gc-sections} allows the linker to prune unreferenced input 
at finer-than-default granularity)
and for speed.
Modern toolchains implement a selection of link-time optimisations, 
although invariably by bunding \emph{intermediate code}, 
such as LLVM IR or \textsf{gcc} GIMPLE, 
into the input files and having the linker invoke compiler middle- or back-end phases.

% \paragraph{Communication with the loader}
% communication with loader: custom program headers, arch metadata / ABI extension metadata
%      (i.e. invoking platform-specific optimisations)
%      -- EXAMPLES?
% 
% \paragraph{Communication with run-time tools}
% communication with tools (distinct from instrumentation): 
%       adding metadata like .gnu.build-id, .note.ABI-tag
%       \_r\_debug
%       \_\_gmon\_start\_\_? 
% 
% \paragraph{Configuration for deployment}
% configure for deployment environment: -rpath, .interp?, 
% 
% \paragraph{Dynamic loading}
% dynamic loading (by leaving the dynamic linker resident/active)
% 
% \paragraph{Interprocedural optimisation}
% interprocedural optimisations
% (don't say "whole program")
% 
% code size optimisation (basic library pull-in algorithm; 
%     --gc-sections;
%     
%     )
% 
% spatial locality optimisation

\paragraph{Compiler-level policy} 
Compilers' use of linker mechanisms is constrained and informed
by efficiency and interoperability concerns. 
\emph{Code models}, conventions about 
how near or far a given definition might lie from a reference to it, 
are crucial to compilers in choosing the most efficient addressing mode
during instruction selection, and are standardised for compiler interoperability reasons.
Mismatched code models cause link errors, caused by overflow in a relocated field 
(a referencing instruction unable to encode the address or distance to its referent).
Code models also define ways to achieve \emph{position independence} of shared library
code---meaning the code does not need to be syntactically
fixed up to operate at different addresses. If so, 
the code may be shared across multiple processes at different mapping addresses.\footnote{It says nothing about
whether the code's \emph{semantics} depend on its load address; 
position-independent code is perfectly well able to 
introspect its own load address, or more generally, to branch on pointers.}
Finally, specific languages often define per-platform ABI conventions to
allow different implementations to interoperate.
We call this \emph{federated compilation}, as a strictly stronger 
requirement than separate compilation.
ABI conventions include data representations, calling conventions and the like; they also 
partially specify a mapping from the language features down to linker features,
and include specification of the allowed code models.
Usually a ``platform ABI'' defines these conventions for C,
and separate documents build on these for other languages (notably \Cplusplus{}), 
often in cross-architecture ways.
The linker is oblivious to these conventions per se; 
its behaviour can be specified independently of them. 
However, adequately specifying the entire toolchain entails
specifying them somewhere, and ensuring that compilers adhere to them.
% A linker would be a good place to check ABI conformance of input binaries.

% \paragraph{Code models}
% 
% Although code models are a compiler 
% feature, 
% (EXCEPT for ldata, lbss etc. -- these come from the ``medium'' code model)
% 
% allowable constraints are consequences of the semantics of linking, 
% and 
% 
% It is the linker's job to reject output if a reference cannot be bound,
% or, when generating a dynamic object, if a relocation it generates 
% \emph{might not} be bindable at run time.
% 
% This optimisation creates an obligation on the linker 
% to reject jobs whose output would contain relocations which might overflow.
% 
% This ``might overflow'' is a consequence of linking's semantics:
% dynamic objects are built contiguously in memory,
% but different dynamic objects may be remote in memory.
% (It could also be triggered
% in the static-link case
% by assigning sections to MEMORY that is far apart.
% FIXME: test this.)
% 
% Compilers generate code making certain assumptions
% 
% which are specified by the programmer
% but implemented by the linker.
% For example, 
% 
% Code is compiled for a specific \emph{code model} which brings constraints
% on how far apart in memory different sections may be assumed to be.
% 
% For example, code destined for an executable binary on 64-bit x86 
% is typically compiled with the ``small'' code model, which assumes 
% that all referenced code and data lives in the lower 32 bits of address space.
% Code destined for shared libraries is compiled instead with a ``position-independent''
% code model which allows arbitrary absolute addresses but 
% permits only PC-relative addressing modes.
% 
% Mismatch of code models shows up as inappropriate 
% (unbindable or might-overflow) relocations
% 
% hence allows instructions to use more efficient addressing modes.
% In turn, references outside this 32-bit area are not expressible,
% and the linker must check 
% 
% 
% code references to shared libraries, which are loaded higher in the address space, 
% are achieved via an indirection (the PLT -- IS THIS TRUE even for 64-bit?), 
% and data references are achieved using a technique called copy relocation
% (\S\ref{sec:relocs}).
% By contrast, code destined for shared libraries is compiled with a position-independent 
% code model which performs the data indirection a slower way 
% (via the GOT) because of semantics of symbol interposition in shared libraries.
% 
% Linkers must reject link jobs in which a 32-bit field is 
% not bound within the output object, since this might overflow
% when bound later to a distant definition 
% (hence infamous ``recompile with \textsf{-fPIC}'' messages).
% 
% EXAMPLE of hello and libhello

% After all that, what have we motivated?


\subsection{Where's the specification?}

It seems natural that a formal specification of linker-speak
could be useful: 
\ps{we can say this more strongly - maybe it belongs in the intro, in
  any case}
as a reference document, a reference implementation, a test oracle, 
and a basis for automated reasoning (either about user programs or 
for the creation of verified toolchains).

Usually, a key input to the production of a formal specification is an informal one.
Unlike most programming languages, no ``standard'' definition of linker-speak has ever been undertaken.
Linker behaviour varies by  vendor, by architecture, by ABI and by operating system.
Even fixing these, no one document holds all, or even most, of the relevant information.

Currently, for each variant, the definitive specifications,
(to the extent that they exist)
lie in a various disparate sources, 
a mixture of informal prose documents, 
an even more informal mailing list posts, 
or sometimes pure implementation.

% The combination of linker, 
% architecture details, 
% basic ABI conventions, 
% and various
% platform-specific and/or 
% de-facto extensions (TLS, \Cplusplus{} conventions, ...)
% ensures that the intended 
% behaviour of linker-speak is spread across
% multiple documents
% which need not be mutually consistent.

In our case of System~V-derived linker-speak, 
relevant source documents include the following.

\begin{description}

\item[SVR4 Interface Description]
The AT\&T System~V (R4) Interface Description, volume 3, includes the manual
page for the System~V linker (p346). 
Much of the contents of this volume were superseded by POSIX, 
but POSIX deliberately omits any mention of either a 
batch linker or dynamic linker as separate entities. 
It specifies a C compiler command-line interface, 
and a similar interface for Fortran 77,
and says in both cases that 
``the system conceptually consists of a compiler and link editor''.

\item[Programmer's Guide] Earlier editions of the System~V's Programmer's Guide, volume 5 (Languages and Support Tools), 
includes a detailed, informative description of how compiler and linker interact,
how the linker operates, and a description of the linker script language.
However, this text was removed from R4's edition of the Guide, 
coincident with the introduction of the ELF format.
The linker detail is replaced with a detailed description of the ELF format itself, 
and of dynamic linking, but \emph{without} details of the linker script language
or the operation of batch linking (beyond the scant details in the manual page).\footnote{Appendix B 
does include one new detail: a ``Mapfile'' syntax which offers a simplified subset 
of linker script features, essentially controlling ELF program headers. 
To our knowledge,  modern linkers do not support this language.
This is not to be confused with ``link maps'' output \emph{from} 
linkers, including AT\&T's System~V linker.}

\item[Linker manuals] Successive linkers, notably the GNU linker, offer detailed manuals
and aim to offer backward compatibility with the System~V linker,
so are logical successors, albeit not very normative, 
to the System~V linker description removed for R4.

\item[ABI Specification] These ELF-related sections of the System~V Programmer's Guide, 
along with various other extracts from the System~V Interface Description,
have been continually revised and released separately
in a document called the System V Application Binary Interface \cite{elf-sco-model}.
This specifies extensions to the ELF format, and alludes to aspects of linker behaviour
without attempting to specify it.

\item[ABI Supplements] Per-architecture \emph{ABI supplements} fill in architecture-specific
details. These include vital details about 
what constraints a link job's input and output objects 
must obey, what structures a batch linker must generate
and what structures a dynamic linker must maintain.
(They also include many higher-level conventions, such as calling conventions
and data representation, which are not directly visible to a linker, but 
which are essential to federated compilation.)

\item[Cross-cutting ABI documents] GNU extensions to the System~V ABIs, for all
architectures, are described in the Linux Standard Base (REF). Thread-local storage
extensions, common to multiple vendors, are described in a succession of documents by Sun,
Red Hat, and latterly independent author (Drepper) (REF again).
Aspects of what was originally an IA64-specific \Cplusplus{} ABI document (REF)
have been adopted in much broader contexts concerning stack walking
and exception handling implementations (REF).

\item[Implementation] No document adequately describes certain aspects of linking.
In others, multiple documents describe them in partial, overlapping ways.
In either case, recourse to an implementation is usually necessary.
The following quotation (from a recent GCC mailing list post by Matthijs van Duin)
is typical of this state of affairs.

{\footnotesize\begin{quotation}
I'd finally like to take a moment to say this was hell to figure
out.... this shit is really documented nowhere properly since it
appears to be a magic blend of ARM EABI and IA64 C++ ABI (meaning
neither documentation quite applies) and I had to plow through the
innards of libgcc and libsupc++ to figure out how things *actually*
work. Argh.
\end{quotation}}

\end{description}


\ps{somewhere we should describe the linkers in widespread use and
  their current status?}

% \begin{itemize}
% 
% \item the System~V linker's command-line interface and its various extensions,
% as documented by linker manuals
% by \citet{att_}, \citet{gnu}, Sun and later Oracle \citet{}, 
% and so on.
% 
% \item a linker script language based on that of the AT\&T linker;
% 
% \item the ELF object file format, augmented by a 
% per-ABI collection refinements, 
% each with its origins in one of various 
% 
% for example, common extensions originate from SunOS / Solaris
% 
% \end{itemize}


\subsection{Consequences}

Our survey has motivated several different needs.
\ps{not really ``conseqences'' or ``needs''}

\ps{structurally distinguish between those we address in this paper (the
  first two?) and the future work we enable (all the rest?)}


\paragraph{Intermediate abstractions for linking}
Language semantics are often described 
using some abstract machine interpreters, 
transition systems, rewrite rules, trace validators, and so on.
What intermediate abstractions are useful for describing 
linker features?

\paragraph{Specification of linking}
What should a linker do?
Tackling this question cracks open 
a large subset of the issues we have discussed: 
the semantics of individual linker mechanisms, 
the linker's input and output formats,
the syntax and semantics of linker-speak
in all its various forms, 
and the per-ABI structures that must be maintained
by both batch linkers and dynamic linkers.

\paragraph{Specification of higher-level ABI conventions}
Higher-level conventions on which federated compilation relies, 
such as calling conventions and data representation,
are separable from linking, but 
are an essential part of a full specificaction of any linking-enabled toolchain.

% - federated compilation stuff: ABI compliance specs

\paragraph{Relating source languages to linker abstractions} 
Language implementers need ways to state their assumptions and guarantees
about how they map the source language to linker mechanisms,
and what they require of the linking environment.
This could potentially allow accurate source-level reasoning about 
linker-supplied definitions (like the introspective \textsf{end} symbols, say) 
and linker-invoked features (such as visibility attributes, aliases, etc.).
% 
% some way for source-level reasoning to account also for link context
%      (by only reasoning about binaries? hmm)
%      "Programs making use of linker-supplied definitions 
%       may assume they exist, and with certain semantics, 
%       to which source-only analyses are oblivious."
% 
% - some way for source lang implementers to state how they map language
%         features onto linker features,
%         and also what they assume about that environment;

\paragraph{Program analyses accounting for linker-speak}
Precise reasoning about composed programs cannot be done entirely at source level; 
linker mechanisms' semantics must be taken into account.
This might motivate several approaches: 
mapping source-level properties down to a
machine-level reasoning substrate (like that of \citet{balakrishnan_wysinyx_}),
or performing intermediate-language reasoning
in a fully linking-enabled context (like the recent LLVM project work
to embed LLVM IR into native object files--REF).

\paragraph{User-facing improvements}
Given the potential for link-time interference between user-supplied
and toolchain-required link behaviour (as with our \textsf{-fvisibility} example), 
we would like better ways to factor linker, compiler and library functionality
that avoid these unwanted iteractions.
In general, much of the user-facing complexity of linking
can be argued as unnecessary.
Many of the linker-speak incantations we surveyed 
are implementing simple properties, but are onerous to state. 
Link errors (such as the infamous ``recompile with -fPIC'' when building shared libraries),
and linker behaviour generally, have a habit of mystifying programmers,
even experienced language implementers.\footnote{Two 
anecdotes in this space include GHC bug 8935,
in which GHC developers grappled with surprising but completely standard
and explainable semantics of \textsf{dlsym()} in shared libraries, 
or OCaml Mantis issue 6462 which incorrectly blames a program corruption bug on a lack of 
renaming or namespacing in the linker.
In fact, the dynamic linker does allow like-named symbols to be defined in multiple 
distinct objects, and this is frequently exploited.}
Weighty documents full of intricate user advice, like Drepper's guide to shared libraries, 
suggest a suboptimal contract between user and toolchain.
Having the user instead merely state high-level policy-like requirements, 
and having a smarter toolchain figure out how to perform the link,
would be far preferable.


% Linkers (and linking-related compiler features) are hard to use,
% and even harder to use optimally.
% 
% Code models are one example
% 
% Drepper's shared library document is another
% 
% (my ``policy list'' is the killer document)



The remainder of this paper describes our work on tackling the first
two of these requirements:
a suitable intermediate abstraction for linking,
and its use in specifying the core linker mechanisms
and related ABI details. Specifically, we describe:

\begin{itemize}

\item a formal specification of the ELF file format;

\item an intermediate representation
       that losslessly abstracts away from ELF 
       to \emph{symbolic partial memory images}, 
       an intermediate representation 
       which captures the operations internal to a linker;

\item an executable specification of a batch linker in terms of our memory image abstraction, 
        capturing the looseness inherent in linking
        and packaged as a post-link validation tool
        (optionally capable of creating output, 
         or enumerating all possible outputs);

\item a formal spec of the ELF extensions adopted by popular toolchains, 
        notably Solaris and GNU;

\item  a semantic definition of ABI-specific structural definitions,
       including how they refine the file format
       and how they construct ABI-specific features during a link.

\end{itemize}

We validate these specifications 
by a series of experiments 
illustrating that 
they cover 
a large sample of real binaries across several architectures (AARCH64, Power64, AMD64, and x86-32)
and incorporating several ABI extensions (GNU? TLS? CHECK),
and that that our notion of link correctness agrees with observable ground truth.
